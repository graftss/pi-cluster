\begin{figure}
\begin{Verbatim}[frame=single]
#include <mpi.h>
#include <stdio.h>
#include <time.h>

#define TRIALS 10000
#define COMM MPI_COMM_WORLD

long long get_microsecond_timestamp() {
  struct timespec ts;
  clock_gettime(CLOCK_REALTIME, &ts);
  return (ts.tv_sec) * 1e6 + ts.tv_nsec / 1e3;
}

void run_root_process(int data_size, int world_size) {
  int data[data_size];
  long long start_time = get_microsecond_timestamp();
  for (int trial = 0; trial < TRIALS; trial++) {
    for (int rank = 1; rank < world_size; rank++) {
      MPI_Send(&data, data_size, MPI_INT, rank, 0, COMM);
    }
  }
  long long duration = get_microsecond_timestamp() - start_time;
  float average_duration = (float)duration / TRIALS;
  printf("payload: %d bytes, avg send time: %.6f microseconds\n",
	 sizeof(int) * data_size, average_duration);
}

void run_nonroot_process(int data_size) {
  int data[data_size];
  for (int trial = 0; trial < TRIALS; trial++) {
    MPI_Recv(&data, data_size, MPI_INT, 0, 0, COMM,
	     MPI_STATUS_IGNORE);
  }
}

int main(int argc, char** argv) {
  int rank, world_size;
  MPI_Init(&argc, &argv);
  MPI_Comm_rank(COMM, &rank);
  if (rank == 0) {
    MPI_Comm_size(COMM, &world_size);
    printf("Using %d \"send\" trials per payload.\n", TRIALS);
    printf("Sharing data between %d processes.\n", world_size);
  }
  for (int size = 1; size < (2 << 12); size *= 2) {
    if (rank == 0) run_root_process(size, world_size);
    else run_nonroot_process(size);
  }
  MPI_Finalize();
  return 0;
}
\end{Verbatim}
    \caption{\texttt{send.c} code listing}
    \label{code:send}
\end{figure}