\begin{figure}
\begin{Verbatim}[frame=single]
#include <stdio.h>
#include <unistd.h>
#include <mpi.h>

void run_scheduler_process(int process_count) {
  int next_server = 1, next_data = 1;
  while (1) {
    MPI_Send(&next_data, 1, MPI_INT, next_server, 
             0, MPI_COMM_WORLD);
    next_data++;
    next_server++;
    if (next_server == process_count) next_server = 1;
    usleep(300000);
  }
}

void run_server_process(int rank, char* hostname) {
  int data, request_complete;
  MPI_Request request;
  while (1) {
    MPI_Irecv(&data, 1, MPI_INT, 0, 0, MPI_COMM_WORLD, &request);
    request_complete = 0;
    while (!request_complete) {
      MPI_Test(&request, &request_complete, MPI_STATUS_IGNORE);
      usleep(300000);
    }
    printf("Server %d received: %d\n", rank, hostname, data);
  }
}

int main(int argc, char** argv) {
  int rank, process_count, hostname_length;
  char hostname[MPI_MAX_PROCESSOR_NAME];

  MPI_Init(&argc, &argv);
  MPI_Comm_size(MPI_COMM_WORLD, &process_count);
  MPI_Comm_rank(MPI_COMM_WORLD, &rank);
  MPI_Get_processor_name(hostname, &hostname_length);

  if (rank == 0) {
    printf("Starting scheduler (rank 0) on %s\n", hostname);
    run_scheduler_process(process_count);
  } else {
    printf("Starting server (rank %d) on %s\n", rank, hostname);
    run_server_process(rank, hostname);
  }

  MPI_Finalize();
  return 0;
}
\end{Verbatim}
    \caption{\texttt{roundrobin.c} code listing}
    \label{code:roundrobin}
\end{figure}